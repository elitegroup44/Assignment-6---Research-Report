\documentclass[12pt,]{article}
\usepackage{zed-csp,graphicx,color}%from

\pagenumbering{roman}
\begin{document}
\begin{titlepage}
 \begin{figure}[h]
  \centerline{\small MAKERERE 
  \includegraphics[width=0.1\textwidth]  {muk_log} UNIVERSITY}
\end{figure}
\centerline{COLLEGE OF COMPUTING AND INFORMATION SCIENCES\\}
\paragraph{}
\centerline{DEPARTMENT OF COMPUTER SCIENCE\\}
\paragraph{}
\centerline{COURSEWORK: RESEARCH METHODOLOGY(BIT 2207)\\}
\paragraph{}
\centerline{LECTURER: MR.ERNEST MWEBAZE}
\paragraph{}
\centerline{\begin{tabular}{|c|c|c|c|}
\hline
\textbf{No.}& \textbf{Student Name} & \textbf{RegNo} & \textbf{Signature} \\ \hline
\textit{1}&\textbf{KIZITO ANDREW} & \textit{16/U/6236/PS}& \textit{} \\ \hline
\textit{2}&\textbf{NAMUYOMBA ANGELLA ZERIDA }& \textit{15/U/10595/PS}& \textit{} \\ \hline
\textit{3}&\textbf{KITIYO MARTIN} & \textit{16/U/18802} & \textit{} \\ \hline
\textit{4}&\textbf{ATUHAIRE DIANA}& \textit{16/U/3826/EVE}  & \textit{} \\ \hline
 \hline
\end{tabular}}

  \begin{flushright}
  The Report,\\
  DATE: $February,6^{th},2018$.
  \end{flushright}
\date{\today}
\end{titlepage}
\newpage

\tableofcontents

\pagenumbering{arabic}
\section{RESEARCH TITLE}
AN ANDROID PHONE APP FOR REDUCING PAPER WORK BY KEEPING TRACK OF PATIENT HISTORY IN ANDROID PHONES\\ FOR NEW LIFE HOSPITAL,MBALE 

\section{abstract}

Smart phones are now ubiquitous. The most popular operating system these days is Android. It's easy to use and affordable to all classes.\\ We can find a lot of applications coming up for android based phones. Games, Security, etc App are available for these phones these days.\\ PATIENT HISTORY TRACKER is an android application. This application will store all the emergency call numbers like Hospital, personal doctor?s number etc. Patient does not need to carry the files of previously held check up. This application provides a log-in for doctor and patient as well. Doctor can access the medical diagnose information of the patient.\\ Patient is registered with a unique ID. So the database of history of a patient is available with all the doctors registered with the application.\\ Patient can view their medical diagnose by logging onto their accounts. When travelling or getting treatment from a new doctor then doctor can easily get the patient?s medical history through this application.\\ Patient can save appointments with Doctor?s and medicine reminders. As per pre-diagnosed diseases, a list of Do?s and Don?ts will be attached to the patients profile. \\These will ease the tedious work of handling all the papers and files of hospitals.\\ It will analyze for common disease being found in most patient and number of people affected. It will provide Medical reports that will be digitized and it will be easy to be handled by one.

\section{Introduction}
\subsection{background}
 "Patient history tracker system" is being developed to assist a patient and doctor can register, view the medical record anywhere, anytime.\\Also reduce the overhead of handling files every time.\\ The main idea of this application is doctor can find the number of people affected by particular disease, anyone can view health tip, Medical record is available all time.\\ This is an innovative application for patient and doctor acceptance that will provide comfort, convenience and efficiency in everyday life. 
\subsection{Motivation:}
 Bringing down health care costs but also facilitate automating patient identification processes in hospitals and use of PDA?s, Smart phones, for design of health care management system.\\the motivation is increasing use of smart phones, so that any users of this system get all service on single click.\\ There are many system developed on restaurant management so to take an idea about all process we reviewed various papers on hospital management,\\ various algorithms and various offline android application and websites which are in market 
\section{Problem Statement:}
There are large number of commercial system applications to allow a single organization to maintain their data related to patient visiting their Hospital for treatment.\\ Basic problems with the existing system are non-interactive environment they provide to users.\\ Use of such system is only beneficial to a single organization.\\ The existing system does not allow Doctor's to find the disease viral and number of people affected.\\ Existing System doesn't record the details of all users.\\ It is heavy to carry all the files (i.e reports and prescription). 
\subsubsection{MAJOR AND MINOR OBJECTIVES:}
\subsubsection{MAJOR OBJECTIVES::}
 To eliminate the burden of carrying medical files including reports and prescription.\\To save time of managing all the tasks for searching reports which are in paper format.\\ To provide user friendly efficient service to all users.\\ To provide reminders for medicine consumptions. 
\subsubsection{MINOR OBJECTIVES:}
To develop an application which  allows patient to view the medical history records online.\\ To reduce the time consumption required for visiting different doctor at different places.\\ To provide user friendly and time saving experience to the patient and doctor as the patients  are able to access their medical history by themselves.
\section{SCOPE OF THE STUDY}
This application use in several Hospitals and Organization. In Future Patient can view his medical reports/history. All in one app. Scope of proposed system is justifiable because large amount of the population face the problem of managing the medical records in form of files and papers.\cite{maloney2007tool}
\section{SIGNIFICANCE OF THE STUDY}
The following are the importance of the study:
i) An automated product which will provides different facilities. 
ii) Patient will get nearest Hospital service.
iii)Automated system for managing different task of provider.
\section{LITERATURE REVIEW:}
There are many system developed on hospital management so to take an idea about all process we reviewed various papers on hospital management, various algorithms and various offline android application and websites which are in market.
\subsection{Medical ID:}
This Application is available for only Ios users. It is offline and saves data for view purpose only including Name, Blood group and basic information about the user available on emergency screen.
\subsection{Smart Hospital based on IOT:}
This paper include scheme of smart hospital based on IOT. In order to overcome to disadvantages of present hospital management system i.e fixed information point, inflexible, etc.
\subsection{Smart hospital management System:}
This paper includes RFID systems integrated with Hospital management system and provides patient identification, staff allocation, Doctors, Medicine, Treatments.
\subsection{NFC based hospital real-time patient management system:}
This paper includes use of NFC (near field communication) Technology can be employed for not only bringing down health care costs but also facilitate automating patient identification processes in hospitals and use of PDA?s, Smart phones, for design of health care management system\cite{nagykaldi2003diabetes}.
\section{METHODOLOGY:}
\subsection{Searching:}
In proposed system we are using Geo-hashing technique for searching. Here, the advantages of this technique over other approaches are:\\
i) With minimum communication and maintenance costs, the underlying data structure can be easily decomposed and shared among a number of co-operating processors, and the technique has been implemented on the Connection Machine\cite{sukovic2009surgical}.\\
 ii) One of the advantages is that geometric hashing is inherently parallel.\\
iii) The structure of geo-hashed data has advantage that data indexed by geohash will have all points for a given rectangular area in contiguous slice.
\section{DETAILED DESIGN}
\subsection{INTRODUCTION:}
System architecture will simplify whole system in such a way that every user of the system gets benefits.\\ As shown in figure there are 3 main users Admin, Doctor and Patient. It?s a tedious job to handle all the medical reports and the generated prescriptions through that. One find it hard to maintain the prescription and reports while travelling or shifting to new place. New doctor need to examine patient overall again to get the exact detail medical condition of the patient.\cite{chapman1987patient}\\ Therefore if the data is digitized and available to Doctor as well as Patient on the tap of their finger then it becomes easy to handle such digitized data.\\ Here the main function is to provide digitized report and prescription to Doctor and Patient and to provide free and online access to the data. The other function is in what pattern user will search the hospital so for that purpose we are using a part of geo-hashing algorithm and GPS system should be on.\\ Person can have facility to search location by location that is home location of the person is detected with GPS and according to selected option location of nearby hospital get searched.\\
Patient registers on the system with detailed information. User ID and password is also provided to the Patient for login.\\ Patient selects the functions to view report or prescription. Doctor is verified by Admin. Doctor also registers and gets provided with Unique user ID and password, which he/she can use for login .\\ Doctor acan view patients profile, Disease Viral, etc.
\section{INTERFACES:}
\paragraph{fig1: Home page }
\includegraphics[width=1.0\textwidth]{./capture2}\\[0.1in]
\paragraph{fig2:Doctors details }
\includegraphics[width=1.0\textwidth]{./capture1}\\[0.1in]
\paragraph{fig3 }
\includegraphics[width=1.0\textwidth]{./capture3}\\[0.1in]
\paragraph{fig4:Patient Details}
\includegraphics[width=1.0\textwidth]{./capture4}\\[0.1in]
\paragraph{fig5:Diagnosisi form }
\includegraphics[width=1.0\textwidth]{./capture5}\\[0.1in]
\paragraph{fig6 }
\includegraphics[width=1.0\textwidth]{./capture6}\\[0.1in]

\section{CONCLUSION:}
Proposed system is based on user need and is user centered. The system will developed in considering all issues related to all user which are included in this system. Wide range of people can use this if they know how to operate android smart phone.\\ Various issues related to hospital management will be solved by providing them a full-fledged system. Thus we are implementing Patient History Tracker system to help and solve one of the problems of people\cite{hawkins1989patient}.



\newpage
\bibliographystyle{IEEEtran}
\bibliography{References.bib}


 \end{document}












